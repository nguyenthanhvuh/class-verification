\documentclass[11pt]{article}
\usepackage[inner=1.5cm,outer=1.5cm,top=2.5cm,bottom=2.5cm]{geometry}
\usepackage{graphicx}
\usepackage{booktabs}
\usepackage[colorlinks,pagebackref,pdfusetitle,urlcolor=blue,citecolor=blue,linkcolor=red,bookmarksnumbered,plainpages=false]{hyperref}

\begin{document}
\begin{center}
{\Large \textsc{CS-695/SWE-699: AI Safety and Assurance}}
\end{center}
\begin{center}
Fall 2023
\end{center}
%\date{September 26, 2014}

\begin{center}
\begin{minipage}[t]{.75\textwidth}
\begin{tabular}{llcccll}
  \toprule
  \textbf{Meetings:} & Tues 4:30PM -- 7:10PM  & & & & \textbf{Place:} & Innovation Hall 215G\\
\textbf{Instructor:} & \href{https://go.gmu.edu/tvn}{ThanhVu Nguyen} & & &  & \textbf{Email:} &  \href{mailto:tvn@gmu.edu}{tvn@gmu.edu}\\
\textbf{Office Hr:} & Tues 11:00AM -- 12:00AM & & & & \textbf{Place:} & ENGR 4430\\
&(email to confirm)&&&&&\\
\bottomrule
\end{tabular}
\end{minipage}
\end{center}


\section{Description}
This special topic course is a research seminar on \textbf{\textbf{AI Verification and Analysis}}.
We will learn various AI verification topics including the applications
of verification, testing, analysis, constraint solving, and abstraction
techniques to Deep Neural Networks such as Feedforward Neural Networks
(FNNs), Residual Networks (ResNet), Convolutional Neural Networks
(CNNs), and Recurrent Neural Networks (RNNs).

The course will focus on active research areas in formal AI reasoning,
but the specific topics will be largely determined by a
combination of instructor fiat and the interests of the students.




\subsection*{Prerequisite}
\begin{itemize}
\item No prerequisite course (but basic knowledge in linear algebra and AI/ML, e.g., CS 580, is strongly recommended)
\item Programming knowledge (Python)
\end{itemize}


\section{Grading}
You will be evaluated based on
\begin{enumerate}
\item Participation:  weekly reading assignments, discussion, and participation (40\%),
\item Programming assignments: 3--4 PA's (40\%), 
\item Project: 1 final project  (20\%)
\end{enumerate}

\subsection*{Group and Submission} 
For your assignments, you can work in \textbf{groups} of 2--3 students.  You can also work by yourself. One member turns in one solution for each assignment. 

Students on a group are expected to participate equally in the effort
    and to be thoroughly familiar with all aspects of the joint work.
    All members bear full responsibility for the completion of
    assignments. 
    Each member receives the same grade for the assignment.  You may change group for different assignments but groups may not be dissolved in the middle of an assignment.
\subsection{Participation}
\label{sec:org16c195f}

On average, we will have \textbf{\textbf{two reading assignments}} each week (about 45 mins for each). 
You are responsible for reading the assignments in advance for any given discussion.  Typically there are two types of reading assignments:  (i) \textbf{chapters} from a neural networks textbook and (ii) from \textbf{research papers}.  

\subsubsection{What to think about for a reading assignment}\label{sec:reading}
\paragraph{Book Chapter} When reading from a book, you should read the assigned chapters carefully, try the examples by hand (e.g., on a piece of paper), and think about the following:
\begin{enumerate}
    \item  \textbf{What is the problem?} (what is the problem the chapter is devoted to? why is it interesting?)
    \item \textbf{What is the solution?} (what is the proposed solution? what are its strengths and weaknesses?)
    \item  provide \textbf{your thoughts} on the reading (e.g., what you like, what you don't like, is it clear?)

\end{enumerate}    
\paragraph{Research Paper} When reading a paper, you should focus on the following: 
\begin{enumerate}
    \item the \textbf{problem} (what is the problem we are trying to solve? why is it interesting?)
    \item the limitations of the \textbf{state of the art} (what are existing approaches? what are their limitations?)
    \item the proposed \textbf{approach} (its novelty, strength, and how it  addresses the weaknesses of  existing work). Typically a paper has an illustrating example, you \emph{should understand it in detail}.
    \item the \textbf{evaluation} and comparison with other approaches (what are the results of the work, how was it evaluated and compared to others?)
    \item provide \textbf{your thoughts} on the paper (e.g., what you like, what you don't like, what you think is interesting, etc.)
    \item \textbf{\textbf{Tools}}: Usually each research paper has a free implementation
tool. I will give bonus points if you try out the
tool and discuss some interesting things that it can or cannot do
(e.g., try the the tool on some small but nontrivial examples). This
will help you understand the readings better and give you ideas on
how to use existing tools in your own work.
\end{enumerate}
    
\subsubsection{Lead Groups} Each reading will be assigned to a group.  That group will lead the class discussion on that reading. The group will be responsible for the following for each reading:
\begin{itemize}
\item You will present \textbf{\textbf{in depth}} the assigned reading to the class (about 25 mins). You can use slides or whiteboard.
\begin{itemize}
    \item Your presentation should answer the questions given in Section~\ref{sec:reading}.
    \item In addition, if the reading has examples (e.g., book chapters have various examples and research papers often have working examples or small code illustration), then your presentation must include and explain those in detail.
\end{itemize}
\item You will guide the discussion (~15 mins), e.g., ask questions to the class, engage others to ask questions and participate in discussion.


\end{itemize}

\subsubsection{Reading Summaries} 

If your group is \emph{not assigned} to lead a reading assignment, then your group will write a 1-page summary of the assignment (so 2 pages for 2 reading assignments). You will submit it to me on Piazza \textbf{before} the day of class that we will discuss the paper (i.e., by 4:29 PM Tuesday).  For the summary, you should answer the questions given in Section~\ref{sec:reading}.

The goals of this approach are to encourage all participants to read the material thoroughly in advance, to provide jumping-off points for detailed discussions, and to allow me to evaluate participation.

\subsection{Programming Assignments (PA's)}

This course consists of several Programming Assignments (PA's) in Python. These PAs are designed for you to gain fundamental knowledge of state of the art AI analysis. All assignments have similar grading weights.

Your submissions will be evaluated for correctness,
organization, and documentation. We will not attempt to fix broken
submissions that fail to execute properly; only limited partial credit will be given in such situations. Assignments are due at \textbf{11:59pm} on the due date.

    

\subsection{Project}
\label{sec:org6da900e}

You (your group) will understand in depth (i.e., \emph{own}) a DNN analysis technique from a research paper. A list of selected papers will be provided and your group will pick one of them (first come first serve).  Your project consists of two parts:

\subsubsection{Example Illustration}
\label{sec:orga9923fc}
\begin{itemize}
\item You will write a ``blog'' describing full concrete example illustrating the DNN technique assigned to you. This blog will be \textbf{due \textbf{2 weeks}} after your group is assigned the paper.

\item Format
\begin{itemize}
\item Written in \textbf{Markdown}
\item Posted on the class's \textbf{Wiki}
\item Consist of a full illustration on how the technique works on a given example (e.g., how Reluplex works on a simple DNN).  
\end{itemize}
\end{itemize}


\subsubsection{Implementation}
\label{sec:orgb697dba}

You will \textbf{implement} the DNN analysis technique using Python. In your blog entry above, you will \textbf{write}  how you apply your implementation to the example illustration you had. The project is \textbf{due on the last day class} (Sat, Dec 2nd).


\section{Honor Code}
\label{sec:orgf24f4c5}

As with all GMU courses, this class governed by the \href{http://oai.gmu.edu/the-mason-honor-code/}{GMU Honor Code}. In this course, all assignments carry with them an implicit statement that it is the sole work of the author.

\section{Learning Disabilities}
\label{sec:org28deb33}

Students with learning disabilities (or other conditions documented with GMU Office of Disability Services) who need academic accommodations should see me and contact the \href{http://ods.gmu.edu/}{Disability Resource Center} (DRC) at (703) 993-2474. I am more than happy to assist you, but all academic accommodations must be arranged through the DRC.


% \section{Links}
% \label{sec:org63a886d}
% \begin{itemize}
% \item \href{assignments.org}{Schedule and Assignments}
% \item \href{project.org}{Project Info}
% \end{itemize}

% \subsection{Related courses}
% \begin{itemize}
% \item \href{https://www.sri.inf.ethz.ch/teaching/reliableai21}{Eth Zurich Reliable and Trustworthy Artificial  Intelligence}
% \end{itemize}

\end{document} 
